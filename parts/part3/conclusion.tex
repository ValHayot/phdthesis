\chapter{Conclusion}

Our implementations significantly reduce processing time by masking data transfers within task computations.
While they can be applied in all cases, they have been found to be most effective when data and computation are
similar. Furthermore, we have not found either implementations to incur significant overheads.

One of the most important characteristics of our implementations is that they do not disrupt the methods in which pipelines
are currently built. Rolling Prefetch allows users to continue writing pipelines in Python as they would normally, maintaining
even the interface for interacting with S3, by being an extension above S3FS. Sea satisfies that requirement by being a preloaded
library that intercepts libc calls. Therefore, rather than adapt their code, users simply need to execute Sea alongside their pipeline.

As more and more researchers are starting to interact with diverse types of infrastructures, such as both cloud and HCP,
it will be important to extend Sea to include functionality to interact with cloud-based storage. It could also be done In
such a way that enable existing pipelines making POSIX calls read from file systems such as 
S3 without instrumentation.



