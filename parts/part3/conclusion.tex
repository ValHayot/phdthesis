\chapter{Conclusion}

In this thesis, we discussed the need for data management strategies in
neuroimaging pipelines to support the ever-growing influx of neuroimaging data.
While many Big Data frameworks have been available for quite some time,
neuroimaging applications are slow to adopt them because of difficulties
employing them on infrastructure in which the pipelines are executed and having
to rewrite standard applications entirely. Furthermore, Big Data Frameworks make
lack some of the qualities that are highly regarded in scientific computing.

We first explored the benefit that a data-intensive neuroimaging pipeline could gain
from employing these strategies and found that it was quite significant. We then looked
at how to adapt frameworks like Apache Spark to make their jobs easier to schedule on HPC
and found that the implemented adaptations did not improve scheduling performance. We also
investigated persistent memory storage as an alternative to adopting software solutions.
While we found the hardware to provide similar performance, we found it was unlikely for it
to be made widely available to researchers in the near future.

Given these results, we decided the best way to provide data-management
strategies to neuroimaging pipelines was to implement our own solutions. We had
two main criteria to consider when designing the implementations: 1) the
solutions had to be lightweight and 2) they did not alter or disrupt in any way
the application features. Our first implementation, Rolling Prefetch, was to
build a tool that minimizes application read time on the cloud by prefetching
the data. Rolling Prefetch is an extension to the S3Fs library that allows
Python applications to directly interact with S3 object stores. By extending an
existing solution, Python applications that already interact with S3 need to be
minimally adapted to function with Rolling Prefetch.

Our second implementation was Sea, a lightweight data-management library for
applications running in Linux environments. Sea is an external tool to the
application, management data solely through the interception of application I/O
calls. By adopting this approach, we were able to ensure that the underlying
application would be able to function as normal without needed to make any
changes.

Our implementations were found to significantly reduce processing time by masking data
transfers within task computations or minimizing data transfers altogether.
While they can be applied in all cases, they
have been found to be most effective when data and computation are similar. In the case of Sea in-memory,
performance can be even better when data transfer time far exceeds computation time.
Furthermore, we have not found either implementations to incur significant
overheads.

As more and more researchers are starting to interact with diverse types of
infrastructures, such as both cloud and HCP, it will be important to extend Sea
to include functionality to interact with cloud-based storage. It could also be
done in such a way that enable existing pipelines making POSIX calls read from
file systems such as S3 without instrumentation.



