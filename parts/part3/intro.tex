\part{Implementing Data Managment Tools for Neuroimaging pipelines}
\section*{Preface}

Neuroimaging pipelines processing large amounts of data may inevitably incur
additional data management penalties. Use of existing Big Data software
solutions require pipeline creators to entirely rewrite well-recognized
neuroimaging tools in order to benefit from the built-in data management made
available by these frameworks. Even if pipelines were to be entirely
reimplemented, performance in HPC environments may still be hindered by the
large resource requests. While hardware solutions, such as Burst Buffers, exist
to alleviate the impacts of data transfers and reduce burden on the parallel
network-based file systems, users are left to perform the data management
themselves. The following chapters focus on two different implementation of two
tools that facilitate data management for neuroimaging pipelines. 

Both implementations focus on the use of an on-the-fly burst buffer. In
Chapter~\ref{chp:rp}, we discuss our implementation and evaluation of Rolling
Prefetch, an extension to the S3Fs Python library that allows users to prefetch
data during data processing. The implementation itself is solely focused on
minimizing the costs of reads from accessing data located from S3 as the final
output data is significantly smaller. We evaluate the performance improvement
relative to the available strategies found in S3Fs. 

Chapter~\ref{chapter:sea-comp} discusses the implementation of Sea, a C/C++ data
management library. The aim of Sea is to bring data management strategies used
in Big Data Frameworks, such as data locality and in-memory computing, to
neuroimaging pipelines without requiring a rewrite of the applications. It uses
the knowledge derived from Chapter~\ref{chp:rp} to implement prefetching,
eviction and flushing for neuroimaging applications running on HPC. Furthermore,
Sea automates within-workflow data management by redirecting application I/O
calls to local caches, whenever applicable. 

We complete our evaluation of Sea in Chapter~\ref{chapter:sea-neuro}, by
benchmarking Sea's performance against that of Lustre's on 3 of the most
popularly used fMRI preprocessing workflows.

While neither of the aforementioned implementations exhibit any characteristics
that limit them to be solely-used in neuroimaging, the are implemented in a way
that keeps the neuroimaging context in mind. For instance, the solution is not
to use Big Data Frameworks because neuroimaging tools would need to be
rewritten. We avoid the use of Big Data filesystems, such as HDFS, as they are
not realistic to HPC environments where access to compute nodes is restricted.
Any pre-existing file system which requires root permissions to be installed is
avoided as we want our tools to be user-installable. Furthermore, user-space
file systems which can exhibit significant overheads, such as FUSE filesystems,
are equally avoided as we do not want the overheads to outweight any possible
benefits.