\part{Evaluation of Sea for Big Data Neuroimaging}
\section*{Preface}

Sea enables researchers to get on-the-fly automated data-management for their
applications running in HPC environments. We have previously seen the speedups that
can be obtained with a synthetic data-intensive pipeline, but it remains to be evaluated
against standard neuroimaging pipelines. Furthermore, we have not yet evaluated how see affects
the four axes of neuroimaging pipelines, namely: 1) ease-of-use, 2) reproducibility,
3) portability and 4) parallelism.

In the following two chapters, we will analyze how using Sea alongside standard neuroimaging
pipelines alters the four axes and how Sea affects the performance of standard neuroimaging pipelines.
The first chapter will be a breakdown of each axis and how Sea meets the requirements, concluding
on a discussion on which how Sea can be improved to further meet the requirements. The second chapter
will consist of a manuscript on a performance analysis of Sea used alongside standard fMRI pipelines
preprocessing representative datasets. The performance analysis experiments are executed in both a controlled and
real HPC environment.