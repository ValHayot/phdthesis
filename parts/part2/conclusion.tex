\chapter{Conclusion}


%% need to include numbers
Our evaluations demonstrate that available Big Data solutions would address performance limitations
arising from increased dataset size. These can be quite high depending on where the pipeline ranks in
terms of the data-to-compute ratio. When comparing Big Data Frameworks to existing neuroimaging pipelines,
it is observed that they perform identically when the built-in data management strategies are removed. Therefore,
neuroimaging pipelines benefit from incorportating such strategies regardless of where they rank currently
within the data-compute-ratio. However, the barrier to entry of using these engines remain high. Widely recognized
neuroimaging pipelines would need to be rewritten for these frameworks and some features standard in neuroimaging pipeline
engines, such as provenance tracking would need to be implemented for these frameworks.

Barrier-to-entry constraints aside, Big Data Frameworks are not well-suited to be executed on HPC clusters. To communicate
between nodes, these frameworks rely on the use of overlay clusters that rely on the availability of multiple nodes for optimal performance.
While our research demonstrates that there is little difference in the time spent on the resource allocation queue between batch and pilot scheduling,
pilot scheduling of Big Data Frameworks are more prone to failures with workers not starting, and batch scheduling of
large numbers of nodes may still increase time spent on the resource allocation queue. 

Software solutions for the handling of Big Data in neuroimaging have reduced applicability to neuroimaging due to the constraints
defined by the constituents of neuroimaging pipelines and the infrastructure that these pipelines are typically executed on. Hardware
solutions such as the \optane can be made available to researchers through access to such specialized infrastructure. With access to such hardware,
neuroimaging pipeline would benefit from speedups without any modification. However, such storage comes at an increased cost and it is likely that 
access would be restricted and potentially temporary. Researchers would still need to perform some form of data management to transfer the data to slower
storage with persistent access.

