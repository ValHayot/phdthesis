\part{Evaluation of Big Data Solutions for Neuroimaging Data Processing}~\label{part:evaluation}
\section*{Preface}
Use and implementation of Big Data Frameworks and strategies is limited in
neuroimaging data processing pipelines. In Part~\ref{chp:background} we discussed
that important attributes to neuroimaging frameworks are centered on aspects
such as ease of use, portability, reproducibility and parallelism. Runtime performance
is entirely abstracted into the parallelism category in which CPU runtime is minimized
by maximizing on CPU usage. Given that neuroimaging pipeline may now be running an unprecedented
amount of data, runtime may be further minimized by reducing the costs associated with data
transfers.

In the following chapters, we investigate the applicability of existing data management
solutions for the processing of neuroimaging data on HPC clusters. The research outlined
in these chapters sought to determine the potential runtime performance
improvement in the processing of large neuroimaging datasets using the solutions currently
available. Should any performance improvement be detected, tools would need to be developed
to help transition existing pipelines to leverage these solutions.

Each chapter consists of a manuscript analyzing different axis of neuroimaging data processing
using Big Data solutions. The three axes selected for our analyses include 1) speedups obtained
from available software, 2) scheduling bottlenecks arising from large resource requests on HPC systems, and
3) speedups obtained from hardware solutions. A simplified performance model is included in 
each of the evaluations to aid in the quantification of the performance bounds.
In Chapter~\ref{chp:bigdatastrategies},
we investigate the potential speedups that can be obtained from using the Apache Spark
framework on a Big neuroimaging dataset (the Big Brain). In Chapter~\ref{chp:spa}, 
we analysed different overlay-scheduling strategies to determine if there was a 
preferential strategy for scheduling Apache Spark applications on HPC. Chapter~\ref{chp:optane}
then investigates the performance speedups related to the processing of the Big Brain using
Intel DC Optane Persistant Memory and how it can work alongside or independently to Big Data
Frameworks.  